%----------------------------------------------------------------------------------------
%	SECTION TITLE
%----------------------------------------------------------------------------------------

\cvsection{Experience}
%----------------------------------------------------------------------------------------
%	SECTION CONTENT
%----------------------------------------------------------------------------------------
\newcommand{\plus}{$+~$~~~~}
\newcommand{\pplus}{$++~$~~~~}
\newcommand{\ppplus}{$+++$~~~~~}
\begin{cventries}
\cventry
{+ ~ First-steps experience ~~~  ~~~ ++ ~ Frequent hands-on experience ~~~  ~~~ +++ ~Lots of experience \vspace{0.1em}} % Job title
{} % Organization
{} % Location
{} % Date(s)
{ % Description(s) of tasks/responsibilities
%\centering
\hspace{1em}
\setlength{\tabcolsep}{3pt}
\begin{tabular}{  l l @{\hskip 3mm} l l }
		   \ppplus & AWS CloudFormation and terraform &  \ppplus & Docker\\
		   \ppplus & k8s: cluster setup(EKS/kops), CI/CD, helm & \ppplus & AWS ECS\\
		   \ppplus & CI/CD, Spinnaker, Jenkins, Bitbucket Pipelines & \ppplus & microservices \\
           \ppplus & java, spring-*, hibernate, maven  & \ppplus & groovy, grails <= 3.*, gradle \\ 
           \ppplus & git & \ppplus & unit/integration testing, TDD/BDD   \\
		   \ppplus & AWS Kinesis, ActiveMQ & \ppplus & postgresql\\
           \ppplus & bash\vspace{0.7em} & \ppplus & linux, macOS, Windows\\ 
           \ppplus & AWS: Lambda, ElasticSearch, Permission Mgmt, RDS, EC2, ALB and ASG & \pplus &python, numpy, scipy, ruby \\ 
           \pplus & mysql, mongodb, influxdb  & \pplus & .net  \\
           \pplus & C\# & \pplus & puppet \\
           \plus & golang, node.js & \plus & Machine Learning 
\end{tabular}
%\begin{table}[]
%\caption{My caption}
%\label{my-label}
%\begin{tabular}{ll}
%+++ & ++
%\end{tabular}
%\end{table}
%\begin{cvitems}
%\item $+++$ 
%\item 
%\item $+++$ 
%\item $+++$ 
%\item $+++$ unit- and integration testing, TDD
%\item \pluses , spring-web-*
%\item \pluses 
%\item \pluses 
%\item \pluses 
%\item \pluses 
%\item \pluses 
%\end{cvitems}
}

%------------------------------------------------

\end{cventries}